%! Author = Anna-Sophie Zaitsewa
%! Date = 19.01.24

% Preamble
\documentclass[11pt]{article}

% Packages
\usepackage{amsmath}
\usepackage{hyperref}
\usepackage{amsfonts}
\usepackage{amssymb}

% Document
\begin{document}
    \begin{titlepage}
        \hspace{3cm}
        \center\large\textsc{
            Complex multiplication of elliptic curves and Hasse-Weil $L$-function:
            Machine-learning approach to analyzing relations}
        \vspace{1cm}
        \center\textsc{Anna-Sophie Zaitsewa}
        \center\small
        \href{mailto:anna.zaitsewa@pm.me}{anna.zaitsewa@pm.me}
        \center\textsc{Warszawa, Polska, 2024}
        \author{Anna-Sophie Zaitsewa}
    \end{titlepage}
    \newpage
    \tableofcontents
    \newpage
    \section{Abstract}\label{sec:abstract}
    This paper is focused on exploring relation between Elliptic Curve's Hasse-Weil $L$-function
    coefficients and CM(Complex-Multiplication) property.
    The machine learning approach was used to analyze $L$-function coefficients.
    Results have shown a possible relation between zeroes of $L$-function coefficients indexed by Sophie-Germain primes
    and CM property of related elliptic curve.

    \section{Introduction}\label{sec:introduction}
    In this section basic definitions and formulas provided.
    \subsection{Elliptic curve}\label{subsec:ellpiticcurve}
    An elliptic curve is a curve given by following equation over a field $K$:
    \[
     y^2 + a_1xy+a_3y=x^3 + a_2x^2 + a_4x + a_5
    \]
    where $a_1, a_2, a_3, a_4, a_5 \in K$

    The simplification of this equation is widely-known Weirstrass form:
    \[
        y^2 = x^3 + Ax + B
    \]
    where $A,B\in K$

    The curve is called singular if a curve has point at which it is not smooth.
    his occurs when the curve either has a cusp (a point with a discontinuous derivative) or a self-intersection.
    In order for elliptic curve to be non-singular its' discriminant should be non-zero:
    \[
        \Delta = -16(4a^3+27b^2) \ne 0
    \]
    
    \subsection{Complex multiplication}\label{subsec:complexmultiplication}


    Isgoeny of elliptic curve $E_1$ to elliptic curve $E_2$ is a morphism $\phi: E_1\to E_2$
    \footnote{Here $E_1$ and $E_2$ are used in terms of a point set of respective curve}, such that
    $\phi(O_{E_1}) = O_{E_2}$.
    As isogeny is a morphism the following property is satisfied: $\forall P,Q\in E_1:\phi(P+Q)=\phi(P)+\phi(Q)$

    A set of all isogenies form homomorphism ring which is denoted as $Hom_k(E_1, E_2)$.

    The endomorphisms of an elliptic curve is defined as follows:
    \[
        End_K(E) \stackrel{def}{=} Hom_K(E,E)
    \]

    The curve $E$ has complex multiplication if and only if its endomorpnisms ring is bigger then set of integers.

    \subsection{Hasse-Weil $L$-function}\label{subsec:hasseweil_l_function}

    The Hasse-Weil $L$-function is defined as follows:
    \[
        L(E,s) = \prod_p L_p(E,s)^{-1}
    \]

    Where $p\in\mathcal{P}$, $\mathcal{P}$ is set of all primes.

    For prime $p$ factor of Hasse-Weil $L$-function is as follows:
    \[
        L_p(s,E)=
        \begin{cases}
            1-a_pp^{-s}+p^{1-2s}, \text{if } p \nmid N\\
            1-a_pp^{-s}, \text{if } p | N, p^2 \nmid N\\
            1, \text{if }p^2 | N
        \end{cases}
    \]

    The coefficient $a_p$ is as follows:

    $a_p = p+1-\#E(\mathbb{GF}_p)$ or $a_p = \pm 1$(for multiplicative reduction)

    Where $\mathbb{GF}_p$ is Galois field of size $p$.

    \section{Objective}\label{sec:objective}
    Our main objective in this paper is to explore possible relations between vector of coefficients $a_p$ and
    property of complex multiplication.
    It is expected that the coefficients would reflect the algebraic properties.
    This paper is focused on exploring relation between particularly Hasse-Weil $L$-function coefficients and complex
    multiplication property without.

    \section{Data}\label{sec:data}
    As a primary data source LMFDB was used.
    With use of SageMath there were found all elliptic curves possessing property of complex multiplication.
    There were 2670 of such curves.
    In order to keep dataset balance additional 2330 random elliptic curves
    were selected.
    Then for each curve coefficients $a_p$ were computed for $p<10000$.

    For computation simplification the coefficients were calculated with procedure used for modular forms as in this paper
    we consider
    elliptic curves over rational fields which makes Taniyama-Weil-Shimura\footnote{Taniyama-Weil-Shimura theorem
    states that any elliptic curve over rational field is a modular form. Which allows us to use algorithm } theorem
    applicable to such curves.

    \section{Methodology}\label{sec:methodology}

    \section{Acknowledgments}\label{sec:acknowlodgements}
    I want to thank K.S.Sorokin for suggesting and helping with this research in 2021.



\end{document}